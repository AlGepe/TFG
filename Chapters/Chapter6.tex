\chapter{Simulation vs Reality: are TRACS results real?}

It has already been discussed the theoretical basis upon which TRACS has been developed to include radiation damage effects into its simulations. It has also been discussed how those implementations were made. Now all that is left for TRACS to be considered a useful tool in research. Just like any tool to be used in research, TRACS needs to be consistent with experimental results or, as we are talking about a simulation tool, be able to reproduce reliably experimental results.

TRACS is able to reprocude any kind of TCT measurement provided an accurate input of carrier distribution inside the detector. To perfom the comparison it is needed to chose one of the three previously mentioned TCT techniques. The SSD group at CERN has in its facilities setups that allow researchers to perform both red-TCT and edge-TCT mesurements with a good level of automation. Due to availability of measuring schedule, the choice was made to perfom edge-TCT measurements in the TCT+ setup at the aforementioned SSD group at CERN, to be compared with simulations from TRACS.

The detector used was a diode that had been previously irradiated with a fluence of $\PHI = 1 \cdot 10^{15} neq$.  The diode has a standard geometry of 1cmx1cm surface with 300 $\mu$m width with a structure similar to that explained in section \ref{REFERENCIA A DEFINIR}. The diode was also subject to the standard annealing process after irradiation and stored in a fridge to ensure constant low temperatures.

In the following sections and extended description of the experimental setup and  measurement procedure will be presented, followed by a discussion of the results and the comparison between TRACS simulations and laboratory measurements.

\section{Experimental procedure} % (fold)
\label{sec:future_improvements}

For a silicon detector to be used as so, it is need to have a circuit that can apply the necessary \vias and read the signal as well as a mechanism to generate free carriers inside the silicon bulk (e.g.: particles or laser). In this section we will explain in detail how these things are set-up together for performing controlled, reliable measurements in the laboratory.

\subsection{Setup} 

Starting with the components of the circuit that are required, one should use a power supply for applying the \vias, a Resistor-Capacitor system for signal filtering and an amplifier for improving Signal to Noise ratio. On top of those components one should, obviously, have a measuring device, often realised in the form of an Oscilloscope.

The RC system and the Amplifier should be placed right after the detector (following the direction of flow of the current) for them to be useful. The Oscilloscope or measuring device should come at the end, as shown in the sketch below this lines REFERENCE HERE TO IMAGE! Even though these placements decisions are fairly trivial it is important to understand what the role of each component in the circuit is.

By looking at the figure REFERENCE AGAIN!!!! it is easier to understand what each of the components is doing and why it is so important to have these as a basic setup. The purpose of having an RC device is to separate the DC component (\vias) from the AC component (signal from the detector) of the total signal that would be otherwise collected at the readout point. By using a Resistor-Capacitor system the \vias and the signal gets separated allowing for the use of an amplifier that would be effective, since \vias is typically over several hundred volts and the amplifiers used do not accept more than 5V. This separation not only allows for the small signal to be amplified but also allows for better S/N ratios 

The reason why the RC system and the amplifier appear boxed together in the sketch is that they usually are part of the same device called 'Bias Tee' due to its shape and the fact that is used for applying bias voltages on silicon detectors. From the amplifier the signal will be read by an oscilloscope connected to it. The oscilloscope is usually set to average a couple hundred signals for smoother and cleaner waveforms as well as to eliminate random fluctuations.The signal is then stored and analysed with the corresponding software, depending on the parameters of interest.

Once we have discussed how the circuit is setup for measurements it is time to talk about some precautions that must be taking into account to ensure the replicability and reliability of the results; mainly related to shielding and isolation. These precautions are standard in any TCT measurement and when they are correctly carried out, they ensure the signal that is been read is the signal coming only from the detector and only due to the carriers created by the laser illuminating the detector.

Isolation of the system from outside sources of signal is performed in two main way: isolation from external light sources and isolation from electromagnetic external signals. For the first of the two tasks it suffices to work in a dark environment. Since electron-hole pairs in silicon are created only by photons in the near infrared or higher energies, it is only necessary that the visible and near infrared be blocked to ensure that all the carriers generated in the silicon come only from the light of the laser used in the experiment.

To insulate the setup from any electromagnetic signals coming from outside sources the most common and widespread solution is to enclose the laser and detector inside a Faraday cage. Faraday cages are easy and cheap to build and provide the needed electromagnetic insulation to ensure the signal read in the experiments is coming only form detector and has no interference (noise) component or (in the real case) as little as possible. Since cable need to be passed through the Faraday cage, it is never possible to build a perfect Faraday cage and have total insulation. In the real case cables coming in and out of the box acting as a Faraday cage need to be properly insulated, to ensure enough EM insulation.

To creat signal inside the detector illumination is required and it is provided by means of a laser, as explained before. Laser provide and easily controllable light source that has a very repeatable behaviour, ideal fro laboratory experiments. The laser used in this particular experiment is a pulsed laser with period between pulses of 240 femtoseconds. This particular laser had a wavelength of 1300nm with an equivalent energy per photon of ~0.95 eV. The low energy photons fulfil the previously stated condition for TPA absorption ($E_{gap} > E_{photon} > \frac{1}{2} E_{gap}$ ) and together with the small period of the pulses ensure two photon process are able to generate charge carrier pairs inside the silicon detector.

When using a low energy laser for TPA absorption, focusing is very important as it will allow for very small regions of space to have enough light intensity for two photon processes to take place. With such a focused laser beam one can control with very fine detaul where to created the charge carriers to sample different points in the detector. Such a fine control requires very precise servo controllers with precision of the order or microns. Such controllers are standard in edge-TCT measurement but only for 2 directions since that kind of laser illuminations yields no information on the third.

The first idea is commonly to mount the laser system on the servo controllers so that the experiment replicates the mental image in which the detector is fixed in space and the laser scans its volume. In reality such a setup is very unpractical for the laser system is often too complex and big to be mounted into the servo controllers. In fact for TPA-TCT experiment like the one conducted here, the laser is too big to fit inside the Faraday cage and sits outside of it; light is then brought, guided and focused inside by a system of mirrors and lenses, most of which live inside the Faraday cage. For the servo controllers to be correctly utilised, the silicon detector is placed on them and then moved around eliminating the need for powerful and strong servo controllers.

%talk about type of detector

\subsection{Measuring procedure} 


The measurements to which simulations will be compared were performed in Bilbao at Singular Laser Facility of the UPV-EHU university. This facility counts with a femtosecond laser and has been developing a TPA-TCT setup together with the IFCA SILICON DETECTOR GROUP\iffalse NOMBRE DEL GRUPO!!!!\fi. Several proof-of-concept measurements were performed as a test before this experiment was conducted. The procedure selection was based on previous experience and trying to maximise the goodness of the results while still performing a test that could be compared with tests performed using different TCT techniques. The values we will focus on will be total collected charge as a function of \vias and Z-position. Using these values it is also possible to estimate $V_{dep}$ and the width of the depletion region for $\vias $ < $V_{dep}$.

In the measurements the detector was moved vertically so that the laser would perform an scan in the Z coordinate as shown in the picture REFERENCE. This type of scanning measurement is commonly called \textit{Z-scan}. For each position in Z a waveform was recorded and later analysed. Each Z-scan was repeated for different \vias obtaining one Z-scan series of waveforms per \vias that will later be analysed together. 

The setup can be controlled from the outside using remote software such as LabView and the different waveforms can be recorded in a plain text file using a specific format that will make analysis afterwards easier. The analysis consisted on condensing each waveform in one point corresponding to the total charge collected at that Z-point. to obtain the collected charge from the waveform it is only necessary to integrate the current over time. All those points will be then plotted in one curve per \vias applied on the detector. 





\section{Results and comparison} % (fold)
\label{sec:future_projection}

data analysis and plots, all plots 


% section future_projection (end)
