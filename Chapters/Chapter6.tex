\chapter{Simulation vs Reality: are TRACS results real?}

It has already been discussed the theoretical basis upon which TRACS has been developed to include radiation damage effects into its simulations. It has also been discussed how those implementations were made. Now all that is left for TRACS to be considered a useful tool in research. Just like any tool to be used in research, TRACS needs to be consistent with experimental results or, as we are talking about a simulation tool, be able to reproduce reliably experimental results.

The first thing that needed to be decided was the type of measurement to reproduce in TRACS. This was a decision made based on the current focus of the silicon group at IFCA \iffalse No se el nombre oficial \fi on TPA-TCT measurements. TPA-TCT is the newest  of all the TCT techniques discussed previously and will also prove the ability of TRACS to adapt to any kind of technique without much work needed. A simulation like this, if proven to be realistic, could also be used as a device to explore TPA limitations and advantages in a fully controlled environment.

For any simulation TRACS needs to be given the initial charge carriers' positions. For these carrier distributions to be physically meaningful the should replicate real world situations such a particle hits or laser illumination (e.g.: for TCT simulations). To recreate laser illumination for the TPA-TCT simulations mentioned just before, python scripts developed by Pablo de Castro and modified by the author were used. Such scripts replicate carrier generation by laser illumination inside silicon volume and can mimic red-TCT, edge-TCT and TPA-TCT depending on the parameters used. In particular, the TPA-TCT illumination simulation script was used to recreate the carriers generated in laboratory measurements using femtosecond infrared laser to perform TPA-TCT measurements.

In the following sections and extended description of the experimental setup and  measurement procedure will be presented, followed by a discussion of the results and the comparison between TRACS simulations and laboratory measurements.

\section{Experimental procedure} % (fold)
\label{sec:future_improvements}

For a silicon detector to be used as so, it is need to have a circuit that can apply the necessary \vias and read the signal as well as a mechanism to generate free carriers inside the silicon bulk (e.g.: particles or laser). In this section we will explain in detail how these things are set-up together for performing controlled, reliable measurements in the laboratory.

Starting with the components of the circuit that are required, one should use a power supply for applying the \vias, a Resistor-Capacitor system for signal filtering and an amplifier for improving Signal to Noise ratio. On top of those components one should, obviously, have a measuring device, often realised in the form of an Oscilloscope.

The RC system and the Amplifier should be placed right after the detector (following the direction of flow of the current) for them to be useful. The Oscilloscope or measuring device should come at the end, as shown in the sketch below this lines REFERENCE HERE TO IMAGE! Even though these placements decisions are fairly trivial it is important to understand what the role of each component in the circuit is.

By looking at the figure REFERENCE AGAIN!!!! it is easier to understand what each of the components is doing and why it is so important to have these as a basic setup. The purpose of having an RC device is to separate the DC component (\vias) from the AC component (signal from the detector) of the total signal that would be otherwise collected at the readout point. By using a Resistor-Capacitor system the \vias and the signal gets separated allowing for the use of an amplifier that would be effective, since \vias is typically over several hundred volts and the amplifiers used do not accept more than 5V. This separation not only allows for the small signal to be amplified but also allows for better S/N ratios 

The reason why the RC system and the amplifier appear boxed together in the sketch is that they usually are part of the same device called 'Bias Tee' due to its shape and the fact that is used for applying bias voltages on silicon detectors. From the amplifier the signal will be read by an oscilloscope connected to it. The oscilloscope is usually set to average a couple hundred signals for smoother and cleaner waveforms as well as to eliminate random fluctuations.The signal is then stored and analysed with the corresponding software, depending on the parameters of interest.

Once we have discussed how the circuit is setup for measurements it is time to talk about some precautions that must be taking into account to ensure the replicability and reliability of the results; mainly related to shielding and isolation. These precautions are standard in any TCT measurement and when they are correctly carried out, they ensure the signal that is been read is the signal coming only from the detector and only due to the carriers created by the laser illuminating the detector.

Isolation of the system from outside sources of signal is performed in two main way: isolation from external light sources and isolation from electromagenetic external signals. For the first of the two tasks it suffices to work in a dark environment. Since electron-hole pairs in silicon are created only by photons in the near infrared or higher energies, it is only necessary that the visible and near infrared be blocked to ensure that all the carriers generated in the silicon come only from the light of the laser used in the experiment.

To isolate the setup from any electromagnetic signals coming from outside sources the most common and widespread solution is to enclose the laser and detector inside a Faraday cage. Faraday cages are easy and cheap to build and provide the needed electromagnetic isolation to ensure the signal read in the experiments is coming only form detector and has no interference (noise) component or (in the real case) as little as possible. Since cable need to be passed through the Faraday cage, it is never possible to build a perfect Faraday cage and have total isolation. In the real case cables coming in and out of the box acting as a Faraday cage need to be properly isolated, to ensure enough EM isolation.
% here goes Faraday cage and obscure chamber explanation

% Here goes laser

% Here goes laser movement and control


\subsection{Setup} 
\subsection{Measuring procedure} 

circuit details and measurement procedure

% section future_improvements (end)

\section{Results and comparison} % (fold)
\label{sec:future_projection}

data analysis and plots, all plots 


% section future_projection (end)
