\chapter{Simulation vs Reality: are TRACS results real?}

It has already been discussed the theoretical basis upon which TRACS has been developed to include radiation damage effects into its simulations. It has also been discussed how those implementations were made. Now all that is left for TRACS to be considered a useful tool in research. Just like any tool to be used in research, TRACS needs to be consistent with experimental results or, as we are talking about a simulation tool, be able to reproduce reliably experimental results.

TRACS is able to reprocude any kind of TCT measurement provided an accurate input of carrier distribution inside the detector. To perfom the comparison it is needed to chose one of the three previously mentioned TCT techniques. The SSD group at CERN has in its facilities setups that allow researchers to perform both red-TCT and edge-TCT mesurements with a good level of automation. Due to availability of measuring schedule, the choice was made to perfom red-TCT measurements in the TCT+ setup at the aforementioned SSD group at CERN, to be compared with simulations from TRACS. In particular bottom-TCT was chose after several measurements for it was the one that showed more clearly the signs of radiatoin damage in the for of the \textit{Double Peak} as a result of the \textit{Duble Junction} explained in section \ref{sec:rad}. 

It is important to note that TRACS needs to be given between 3 and 7 different parameters to perform a correct simulation of an irradiated silicon detector. This complexity call for automated fitting software to perform parameter selection to match TRACS simulation to the experimental results. Such a software is planned to be developed but it is not available at the time of writting this report. It is for this reason that comparison with real-world measurements will only be performed with one detector and one technique. Selection and optimisation of the previously mentioned parameters that TRACS require to perform a simulation of irradiated silicon detector will be done manually by trial-and-error. Such parameters are not expected to be quantitatively accurate due to the complexity of the fitting process. The simulation-measurement comparison is, hence, not intended as categorical proof of TRACS accuracy but rather as a showcase of TRACS newly implemented characteristics; namely reproduce the features of the transients generated by irradiated silicon detectors.

To avoid adding more sources of complexity to the simulation, the detector used was the most simple silicon detector: a diode. The diode had been previously irradiated with a fluence of $\PHI = 10^{15} neq$. It has a standard diode geometry of 1cmx1cm surface with 300 $\mu$m width with a structure similar to that explained in section \ref{REFERENCIA A DEFINIR}. The diode was also subject to the standard annealing process after irradiation and stored in a fridge to ensure constant low temperatures and prevent radiation effects to grow over time. This also complies with the standard irradiation-storage-measurement procedure that is followed in most of the research experiments on radiation damage in silicon detectors.

In the following sections an extended description of the experimental setup and  measurement procedure will be presented, followed by a discussion of the results and the comparison between TRACS simulations and laboratory measurements.

\section{Experimental procedure} % (fold)
\label{sec:future_improvements}

The basics of TCT measurement techniques have been explained in section \ref{sec:tct} so this section will focus on the more specific implementation of those principles of operation in the TCT+ setup at CERN. The TCT+ setup is a standard laboratory setup belonging to the SSD group in which this project was developed. The measurements were performed by the author using the previously mentioned TCT+ setup under supervision of a member of the SSD group. Therefore the details explained in the following subsections will be specific to the TCT+ setup as used by the SSD group at CERN. However, since most of the components and arrangement of TCT+ are similar to those used in most of the TCT-capable facilites, many of the explanations ahead will be applicable to any TCT setup elsewhere. 

Explanation of the experimental procedure will be presented after setup description. The procedure that will be described also falls under the same considerations as TCT+ setup, since it will also be specific to the particular measurement performed for this project but will be applicable for any TCT measurement almost entirely. 

The process of adjusting the parameters for the simulation to match the measurements will also be commented, albeit briefly. The fitting of the simulations will be performed in a purely trial-an-error manner and is not of muhc interest for the project as software is expected to be available to perform such task in a more efficient manner. 

%To perform the bottom-TCT measurements For a silicon detector to be used as so, it is needed to have a circuit that can apply the necessary \vias and read the signal as well as a mechanism to generate free carriers inside the silicon bulk (e.g.: particles or laser). In this section we will explain in detail how these things are set-up together for performing controlled, reliable measurements in the laboratory.

\subsection{Setup} 

In the first place, the components of the circuit that are required must be stated. The list contains a power supply for applying the \vias, a Resistor-Capacitor system for signal filtering and an amplifier for improving Signal to Noise ratio. An Oscilloscope was also used as measuring device for it has fast response and enough resolution to measure the signal comming from the silicon diode.

The RC system and the Amplifier are placed right after the detector (following the direction of flow of the current). The Oscilloscope or measuring device is positionedat the end, as shown in the sketch below this lines REFERENCE HERE TO IMAGE! Even though these placements decisions are fairly trivial it is important to understand what the role of each component in the circuit is.

By looking at the figure REFERENCE AGAIN!!!! it is easier to understand what each of the components is doing and why it is so important to have such a configuration even for the most basic setups. The purpose of having an RC device is to separate the DC component (\vias) from the AC component (signal from the detector) of the total signal that would be otherwise collected at the readout point. By using a Resistor-Capacitor system the \vias and the signal gets separated allowing for the use of an amplifier that would be effective, since \vias is typically over several hundred volts and the amplifiers used do not accept more than 5V. This separation not only allows for the small signal to be amplified but also allows for better S/N ratios 

The reason why the RC system and the amplifier appear boxed together in the sketch is that they usually are part of the same device called \textit{Bias Tee} (due to its shape and the fact that is used for applying bias voltages on silicon detectors). From the amplifier the signal will be read by the oscilloscope connected to it. The oscilloscope is usually set to average a couple of hundreds of signals for smoother and cleaner waveforms as well as to eliminate random fluctuations.The signal is then stored and analysed with the corresponding software, depending on the parameters of interest.

Once it has been discussed how the circuit is setup for measurements it is time to focus on some consideration that need to be taken into account to ensure the replicability and reliability of the results; mainly related to eliminate noise from the signal. These precautions are standard in any TCT measurement and when they are correctly carried out, they ensure the signal that is read is  only the signal coming from the detector and only due to the carriers created by the laser illuminating the detector.

Isolation of the system from outside sources of signal is performed in two main ways: isolation from external light sources and isolation from electromagnetic external signals. For the first of the two tasks it suffices to work in a dark environment. Since electron-hole pairs in silicon are created only by photons in the near infrared or higher energies, it is only necessary that the visible and near infrared be blocked to ensure that all the carriers generated in the silicon come only from the light of the laser used in the experiment.

To insulate the setup from any electromagnetic signals coming from outside sources the most common and widespread solution is to enclose the laser and detector inside a Faraday cage. Faraday cages are easy and cheap to build and provide the needed electromagnetic insulation to ensure the signal read in the experiments is coming only form detector and has no interference (noise) component or (in the real case) as little as possible. Since cable need to be passed through the Faraday cage, it is never possible to build a perfect Faraday cage and have total insulation. In the real case cables coming in and out of the box acting as a Faraday cage need to be properly insulated, to ensure enough EM insulation.

To create signal inside the detector, laser illumination is used. Laser provide and easily controllable light source that has a very repeatable behaviour, ideal for laboratory experiments. The laser used in this particular experiment is a red laser with wavelength $\lambda = 660nm$ with other TCT techniques require of different wavelengths and laser properties. The asociated photon energy of the red laser used in our experiment ($E_{red} \approx 1.88 eV$) is above the required energy to promote electrons from the valence band to the conduction. Photons coming from the red laser are able to create electron-hole pairs on their own, so charge carriers of both signs will be created inside the detector insofar as the beam can penetrate inside the detector which for red light is of the order of a few $\mu m$, as we have previously mentioned in section \ref{sec:tct}
%When using a low energy laser for TPA absorption, focusing is very important as it will allow for very small regions of space to have enough light intensity for two photon processes to take place. With such a focused laser beam one can control with very fine detaul where to created the charge carriers to sample different points in the detector. Such a fine control requires very precise servo controllers with precision of the order or microns. Such controllers are standard in edge-TCT measurement but only for 2 directions since that kind of laser illuminations yields no information on the third.

The detector is then mounted on motorised servo controllers that can change the detector position with respect to the laser beam with micrometric precision. In TPA and edge TCT measurements these servo controllers are used to effectivelly move the laser inside the detector when performing different types of scans. In the case of red-TCT they are equally important but only used to align the detector window with the laser beam. For bottom-TCT measurements it is not really critical where the laser illuminates the detector as long as it is not reflected on the metal frame. Since the diode has a very simple geometry, almost any part of the detector behaves the same for red illumination. The only requirement for consistency being that the beam should be completely inside the window openned in the metalisation part of the diode specifically for that purpose.

Temperature is very important when measuring silicon detectors, not only because it can accelerate radiation damaging but also because the mobility of electrons and holes is dependant on such magnitude. To make sure the temperature of the silicon detector remains constant throught the experiment, a peltier device is used. Peltier devices exploit Peltier effect to change their temperature at very fast rates. They are commonly used in TCT measurements for both temperature control and temperature change. In the measurement performed for the project it was used for both purposes with a target temperature of $T = -100C \approx 170K$ during all the measurements.

\subsection{Measuring procedure} 

The measuring procedure itself is fairly simple but there are some preparation steps that need to be carried out to ensure good results. Once everything has been set up correctly and re-checked one can proceed to operate. First critical point to cover is to align the laser beem with the window in the diode

% ALL WRONG CAUSE ITS e-TCT AND NO TPA

%The measurements to which simulations will be compared were performed in Bilbao at Singular Laser Facility of the UPV-EHU university. This facility counts with a femtosecond laser and has been developing a TPA-TCT setup together with the IFCA SILICON DETECTOR GROUP\iffalse NOMBRE DEL GRUPO!!!!\fi. Several proof-of-concept measurements were performed as a test before this experiment was conducted. The procedure selection was based on previous experience and trying to maximise the goodness of the results while still performing a test that could be compared with tests performed using different TCT techniques. The values we will focus on will be total collected charge as a function of \vias and Z-position. Using these values it is also possible to estimate $V_{dep}$ and the width of the depletion region for $\vias $ < $V_{dep}$.

%In the measurements the detector was moved vertically so that the laser would perform an scan in the Z coordinate as shown in the picture REFERENCE. This type of scanning measurement is commonly called \textit{Z-scan}. For each position in Z a waveform was recorded and later analysed. Each Z-scan was repeated for different \vias obtaining one Z-scan series of waveforms per \vias that will later be analysed together. 

%The setup can be controlled from the outside using remote software such as LabView and the different waveforms can be recorded in a plain text file using a specific format that will make analysis afterwards easier. The analysis consisted on condensing each waveform in one point corresponding to the total charge collected at that Z-point. to obtain the collected charge from the waveform it is only necessary to integrate the current over time. All those points will be then plotted in one curve per \vias applied on the detector. 





\section{Results and comparison} % (fold)
\label{sec:future_projection}

data analysis and plots, all plots 


% section future_projection (end)
