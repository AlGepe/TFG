\chapter{Simulation vs Reality: are TRACS results real?}

It has already been discussed the theoretical basis upon which TRACS has been developed to include radiation damage effects into its simulations. It has also been discussed how those implementations were made. Now all that is left for TRACS to be considered a useful tool in research. Just like any tool to be used in research, TRACS needs to be consistent with experimental results or, as we are talking about a simulation tool, be able to reproduce reliably experimental results.

The first thing that needed to be determined was the type of measurement to reproduce in TRACS. This was a decision made based on the current focus of the silicon group at IFCA \iffalse No se el nombre oficial \fi on TPA-TCT measurements. TPA-TCT is the newest of all the TCT techniques discussed previously and will also prove the ability of TRACS to adapt to any kind of technique without much work needed. A simulation like this, if proven to be realistic, could also be used as a device to explore TPA limitations and advantages in a fully controlled environment.


\section{Experimental procedure} % (fold)
\label{sec:future_improvements}

circuit details and measurement procedure

% section future_improvements (end)

\section{Results and comparison} % (fold)
\label{sec:future_proyection}

data analysis and plots, all plots 


% section future_proyection (end)
