\chapter{TPA-TCT and other characterisation techniques}

Characterisation of silicon detectors both before and after radiating them is of great importance to understand the effects of radiation in silicon detectors and for trying to counteract them. Typical measurements performed in the laboratories for research purposed include I/V and C/V curves measurements as well as light ilumination using Transient Current Techniques (TCT) in any of its variants. 

Rationale behind the measurements

\section{TCT} % (fold)
\label{sec:experimental_method}

Where how and what we measured. These are not real world measurements but testing/calibrating measurements that can tell us how good the simulation is. 

% section experimental_method (end)o

\section{e-TCT} % (fold)
\label{sec:fitting_method}

Here is where we show the pivoting capabilities of the project where it can be used as a simulator to predict the signal from a very well known detector as well as to "measure"/know the field and space charge distribution inside an irradiated sample of which we only know their transients.

% section fitting_method (end)

\section{TPA} % (fold)
\label{sec:results_and_comparison_with_tracs_rad}

Comparison of the results obtained from the actual measurements and the fitted space charge distribution. These are THE results of the bachelor's thesis. Most important part of the whole  project. We shall be objective and very critic with optimism and without forgetting about the grand scheme of thing.

% section results_and_comparison_with_tracs_rad (end)
