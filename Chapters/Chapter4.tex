\chapter{TPA-TCT and other characterisation techniques}

Characterisation of silicon detectors both before and after radiating them is of great importance to understand the effects of radiation in silicon detectors and for trying to counteract them. Typical measurements performed in the laboratories for research purposed include I/V and C/V curves measurements as well as light ilumination using Transient Current Techniques (TCT) in any of its variants. 

The I/V and C/V measurements are purely electrical measurements that do not measure detector specific parameters. Curves of Intensity versus Voltage and Capacitance versus Voltage are good indicatives of the quality of the silicon detector as well as providing with a very precise measurement of $V_{dep}$ that can be later crossed check with illumination techniques. I/V and C/V measurements require simpler setups that TCT measurements but fall short in detector specific characterisation as they are performed with the silicon element as a pure diode not as a detector.

Illumination techniques are the ones that exploit the particle detection capabilities of the silicon piece. The most common techniques are TCT in which the current is recorded over time and the resulting waveforms are analysed later. This kind of measurements are more similar to real world detector usage and thus yield more relevant information about detector performance and degradation. As with the case of C/V and I/V measurements both can be done before and after irradiation in the same manner.

For the purposes of this work I/V and C/V measurements will not be considered in detail and focus should be places on TCT measurements for those are the type of measurements that TRACS simulation software is able to reproduce. In the following sections we will explain the three main TCT techniques that are currently performed as TCT silicon detector characterisation, namely red (or normal) TCT, edge-TCT and Two Photon Absorption (TPA) TCT. All of them share the same basic principles of illumination and signal recording but differ in key aspects that make them more suitables for characterising one type of detector or measuring one property. 

\section{TCT} % (fold)
\label{sec:experimental_method}

The measuring procedure for all TCT measurements is fairly similar, the main difference being the wavelength of the laser used and how the laser is point at the detector. Since silicon absorption coefficient is dependent on the wavelength of the light it follows that different laser frequencies will have different penetration and will yield substantially different results. All this techniques exploit the fact that photons with enough energy can create electron-hole pair than will then drift inside the detector generating an electrical current. 

The most basic of TCT techniques is the so-called red-TCT (also refered to as top/bottom-TCT depending on where the laser is shot). This kind of TCT measurements consist on illuminating the top or bottom part of the detector with a red laser with a typical wavelength around XXXnm. The absorption of these photons is a first order process since they have enough energy higher than silicon's band gap. Since silicon transmitance for red light is very low, all the photons are absoved withing a few $\mu m$ from the edge. 

All the electron-hole pairs will then be created in a small region very close to on of the collection electrodes. 

specific details of red-TCT

\section{e-TCT} % (fold)
\label{sec:fitting_method}

details of edge TCT

% section fitting_method (end)

\section{TPA} % (fold)
\label{sec:results_and_comparison_with_tracs_rad}

TPA particularities including second order absortion and penetration\ldots and voxel thingies


% section results_and_comparison_with_tracs_rad (end)
