\chapter{TPA-TCT and other characterisation techniques}

Characterisation of silicon detectors both before and after radiating them is of great importance to understand the effects of radiation in silicon detectors and for trying to counteract them. Typical measurements performed in the laboratories for research purposed include I/V and C/V curves measurements as well as light ilumination using Transient Current Techniques (TCT) in any of its variants. 

The I/V and C/V measurements are purely electrical measurements that do not measure detector specific parameters. Curves of Intensity versus Voltage and Capacitance versus Voltage are good indicatives of the quality of the silicon detector as well as providing with a very precise measurement of $V_{dep}$ that can be later crossed check with illumination techniques. I/V and C/V measurements require simpler setups that TCT measurements but fall short in detector specific characterisation as they are performed with the silicon element as a pure diode not as a detector.

Illumination techniques are the ones that exploit the particle detection capabilities of the silicon piece. The most common techniques are TCT in which the current is recorded over time and the resulting waveforms are analysed later. This kind of measurements are more similar to real world detector usage and thus yield more relevant information about detector performance and degradation. As with the case of C/V and I/V measurements both can be done before and after irradiation in the same manner.

For the purposes of this work I/V and C/V measurements will not be considered in detail and focus should be places on TCT measurements for those are the type of measurements that TRACS simulation software is able to reproduce. In the following sections we will explain the three main TCT techniques that are currently performed as TCT silicon detector characterisation, namely red (or normal) TCT, edge-TCT and Two Photon Absorption (TPA) TCT. All of them share the same basic principles of illumination and signal recording but differ in key aspects that make them more suitables for characterising one type of detector or measuring one property. 

Rationale behind the measurements

\section{TCT} % (fold)
\label{sec:experimental_method}

Where how and what we measured. These are not real world measurements but testing/calibrating measurements that can tell us how good the simulation is. 

% section experimental_method (end)o

\section{e-TCT} % (fold)
\label{sec:fitting_method}

Here is where we show the pivoting capabilities of the project where it can be used as a simulator to predict the signal from a very well known detector as well as to "measure"/know the field and space charge distribution inside an irradiated sample of which we only know their transients.

% section fitting_method (end)

\section{TPA} % (fold)
\label{sec:results_and_comparison_with_tracs_rad}

Comparison of the results obtained from the actual measurements and the fitted space charge distribution. These are THE results of the bachelor's thesis. Most important part of the whole  project. We shall be objective and very critic with optimism and without forgetting about the grand scheme of thing.

% section results_and_comparison_with_tracs_rad (end)
