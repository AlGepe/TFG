\chapter{TRACS: the meats and potatoes of this whole thing}

With all the data taken and already compared it's time to look at the grand scheme of things and conclude how good this project is, how good things developed and how good we developed things.

\section{Results and achievements} % (fold)
\label{sec:results_and_achievements}

here we shall talk about the initial goals and how well we accomplished them. Having experimental and simulated measurements is great (+1pt for the project) but we need to analyse how good the simulator is (+-1pt depending on the outcome).

The project deserves between A+ and B+ => This should be the take away of the conclusion

% section results_and_achievements (end)

\section{Let's put the project in perspective} % (fold)
\label{sec:let_s_put_the_project_in_perspective}

It's time to sell the product! It has been presented in RD50 (Hamburg) and validated by the collaboration (at least PdC part). The tool is unique in many ways. From its never-seen-before ability to manipulate and fit the space charge distribution inside a detector to the more common but still in development capacity to simulate and predict transients on radiated silicon detectors.

It's a tool for the future, working in the limits of knowledge and (maybe) pushing them forwards. (toma sobrada!)

% section let_s_put_the_project_in_perspective (end)
