\chapter{Silicon detectors: all you always wanted to know but never dared asking} %PREELIMINARY TITLE % (fold)
\label{cha:simulator_development}

Silicon detectors work using the same basic principles that almost every other particle detector uses. This classical "detector configuration" consists fundamentally in an enclosed volume which is filled with a material in which free charges can be generated by and incident particle. The second part of a typical detector is a potential difference between two points inside the volume, these can also be part of the edge of said volume. The process by which signal is collected is very easy to understand in this general detector.


\section{P-N Junction}

\section{Carrier Transport}


\section{Recombination} 

\section{Carrier Generation}

\section{Signal Generation: Ramo's Theorem} % No es Sergio Ramos, es otro Ramo's

