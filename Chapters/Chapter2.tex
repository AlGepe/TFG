\chapter{Silicon detectors: all you always wanted to know but never dared asking} %PREELIMINARY TITLE % (fold)
\label{cha:simulator_development}

Silicon detectors work using the same basic principles that almost every other particle detector uses. This classical "detector configuration" consists fundamentally in an enclosed volume which is filled with a material in which free charges can be generated by and incident particle. This volumen is then connected to an electrical circuit that creates a  potential difference between two points inside the volume, these can also be part of the edge of said volume. The process by which signal is collected is very easy to understand in this general detector.

When an incoming particle goes through the volumen there's a probability that this particle might create one or more pairs of free charges, one positive and one negative (either electron-ion or electron-hole pair), these charges then move in the electric field inside de volumen following opposite trajectories depending on their charge. The movement of charges creates a current in the circuit by electrical induction that depends on the speed of the charged particles and the total amount of charge. This current created in the circuit is what can be meassured in the lab and analysed to obtain all the relevant physical quantities.

Even though this general concept of particle detector holds true for almost every detector in use nowadays the implementation of this concept varies greatly depending on the material from which the detector is constructed. In the following sections we will see in detail how silicon can be turned into a particle detector as well as the operational principles of silicon detectors

\section{P-N Junction}

Silicon is a semiconductor material and as such it has some interesting properties. In a pure state silicon does not conduct electrity well enough to be a good material to build a detector. However, as any semiconductor its electrical properties can be easily modified by adding controlled impurities a pure silicon crystal; this process is called doping. There are two basic types of doping: n-doping and p-doping which differ one from another in the properties of the impurity atoms.

On the one hand n-doping is produced by introducing atoms with (typically) one more electron in the outter shell than silicon in between normal silicon atoms. These impurities in the form of atoms such as Phosphorus, Arsenic... have little impact in the lattice structure and properties aside from the creation of new energy levels near silicon's (VB)

\section{Carrier Transport}


\section{Recombination} 

\section{Carrier Generation}

\section{Signal Generation: Ramo's Theorem} % No es Sergio Ramos, es otro Ramo's

