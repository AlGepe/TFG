\chapter{Silicon detectors: all you always wanted to know but never dared asking} %PREELIMINARY TITLE % (fold)
\label{cha:simulator_development}

Silicon detectors work using the same basic principles that almost every other particle detector uses. This classical "detector configuration" consists fundamentally in an enclosed volume which is filled with a material in which free charges can be generated by and incident particle. This volume is then connected to an electrical circuit that creates a  potential difference between two points inside the volume, these can also be part of the edge of said volume. The process by which signal is collected is very easy to understand in this general detector.

When an incoming particle goes through the volume there's a probability that this particle might create one or more pairs of free charges, one positive and one negative (either electron-ion or electron-hole pair), these charges then move in the electric field inside the volume following opposite trajectories depending on their charge. The movement of charges creates a current in the circuit by electrical induction that depends on the speed of the charged particles and the total amount of charge. This current created in the circuit is what can be measured in the lab and analysed to obtain all the relevant physical quantities.

Even though this general concept of particle detector holds true for almost every detector in use nowadays the implementation of this concept varies greatly depending on the material from which the detector is constructed. In the following sections we will see in detail how silicon can be turned into a particle detector as well as the operational principles of silicon detectors

\section{P-N Junction}

Silicon is a semiconductor material and as such it has some interesting properties. In a pure state silicon does not conduct electricity well enough to make for a good detector material. However, as any semiconductor its electrical properties can be easily modified by adding controlled impurities a pure silicon crystal; this process is called doping. There are two main types of impurities used for silicon doping: acceptor impurities and donor impurities.

Donor impurities have one electron more in the outer shell than silicon. Donor impurities introduce extra energy levels close to the Conduction band filled with the extra electrons they bring. Conversely acceptor impurities have one less electron in the outer shell producing new empty energy levels close to the Valence band. These last empty energy levels are better pictured in solid state physics as being filled with a hole\footnote{Holes are quasi-particles used in solid state physics as a more convenient way of describing the lack of an electron} since it groups all the effects of the electrons moving in and out of empty energy levels into one particle.

These donor (or acceptor) levels introduced by impurities are very close to the conduction (Valence) band, so much so that energies of the order of $k_B T$ at room temperature are enough to excite electrons into the conduction band (from the valence band into the impurity level). These excitations result in the doped silicon being able to conduct electricity much better than before and allowing silicon to be used as detector material.

Silicon can be doped with donor impurities only, acceptor impurities only or both types of impurities. In the latter case is the difference between the number of donor impurities ($N_D$) and acceptor impurities ($N_A$) dictates whether the doped silicon will be considered $p$-doped ($N_A > N_D $) or n-doped ($N_D > N_A$). When $n$-doped and $p$-doped silicon are put in contact or when there is an abrupt jump in the silicon from $n$-doping to $p$-doping, the so called $p-n \hspace{5pt} junction$ appears and some strange effects appear.

To understand what are those effects and why they happen we should start by looking at each type of silicon on both sides of the junction. On the one hand we have and excess of acceptors $N_A$ ($p$ side) whilst on the other side there is an excess of donors $N_D$ ($n$ side) this produces a gradient of the chemical potential that makes impurities diffuse from one side to the other. This diffusive migration of donor impurities to the $p$ side and acceptors to the $n$ side creates an electrical potential difference due to said movement of charged impurities from one side to the other. The resulting electric field produces the opposite effect in the charged impurities than the chemical potential. Since the ionised atoms in the silicon lattice cannot effectively move from their initial positions, an equilibrium is achieved when the chemical and electrical potentials balance its effect resulting in an steady state. 

In such a equilibrium state 2 zones can be clearly identified near the $p-n$ junctions, on top of the unaffected $p$ and $n$ regions of the silicon far away from the junction. These two zones are electrically charged having an excess of negative charge on the $p$ side and a excess of positive charge on the $n$ side of the junction. In this region, called the space charge region, an electrical field and its correspondent potential can be found. This region is called the depletion region because it has been depleted of free charge carriers.

\section{Effective space charge distribution ($N_{eff}$)} 
%Talk equations and introduce the concept of Neff and its shape in a non-irradiated silicon detector

The electrical properties of the $p-n$ junction can be derived from the corresponding Poisson equation. \[ENTER POISSON EQUATION\]

[PARAGRAPH EXPLAINING VARIABLES]



It is easy to see from equation [POISSON] the important role of the space charge distribution, also called $N_{eff}$. Such quantity will also play a very important in the parametrisation and implementation of radiation damage in the TRACS simulator, as we will explain in detail in sections [SECTION]. For now, we will focus on the non-irradiated case in which the $N_{eff}$ is constant throughout the depleted zone. 
When Poisson's equation is solved for $\rho = const$ the resulting electric field is linear with the distance to the $p-n$ junction.



\section{Carrier Transport}


\section{Recombination} 

\section{Carrier Generation}

\section{Signal Generation: Ramo's Theorem} % No es Sergio Ramos, es otro Ramo's

