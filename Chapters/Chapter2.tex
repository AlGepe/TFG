\chapter{Silicon detectors: all you always wanted to know but never dared asking} %PRELIMINARY TITLE % (fold)
\label{cha:simulator_development}

Silicon detectors work using the same basic principles that almost every other particle detector uses. This classical "detector configuration" consists fundamentally in an enclosed volume which is filled with a material in which free charges can be generated by and incident particle. This volume is then connected to an electrical circuit that creates a  potential difference between two points inside the volume, these can also be part of the edge of said volume. The process by which signal is collected is very easy to understand in this general detector.

When an incoming particle goes through the volume there's a probability that this particle might create one or more pairs of free charges, one positive and one negative (either electron-ion or electron-hole pair), these charges then move in the electric field inside the volume following opposite trajectories depending on their charge. The movement of charges creates a current in the circuit by electrical induction that depends on the speed of the charged particles and the total amount of charge. This current created in the circuit is what can be meassured in the lab and analysed to obtain all the relevant physical quantities.

Even though this general concept of particle detector holds true for almost every detector in use nowadays the implementation of this concept varies greatly depending on the material from which the detector is constructed. In the following sections we will see in detail how silicon can be turned into a particle detector as well as the operational principles of silicon detectors

\section{P-N Junction}

Silicon is a semiconductor material and as such it has some interesting properties. In a pure state silicon does not conduct electricity well enough to be a good material to build a detector. However, as any semiconductor its electrical properties can be easily modified by adding controlled impurities a pure silicon crystal; this process is called doping. There are two main types of impurities used for silicon doping: acceptor impurities and donor impurities.

Donor impurities have one electron more in the outer shell than silicon. Donor impurities introduce extra energy levels close to the Conduction band filled with the extra electrons they bring. Conversely acceptor impurities have one less electron in the outer shell producing new empty energy levels close to the Valence band. This last empty energy levels are better pictured in solid state physics as being filled with a hole\footnote{Holes are quasi-particles used in solid state physics as a more convenient way of describing the lack of an electron}



There are two basic types of doping: n-doping and p-doping which differ one from another in the properties of the impurity atoms.

On the one hand n-doping is produced by introducing atoms with (typically) one more electron in the outer shell than silicon in between normal silicon atoms. These impurities in the form of atoms such as Phosphorus, Arsenic... have little impact in the lattice structure and properties aside from the creation of new energy levels very close to silicon’s conduction band (CB) where the extra electrons sit at \textit{T = 0K}. At higher temperatures (such as room temperature) thermal excitation is enough to promote those doping electrons to the CB which means the material now conducts electricity very well.

On the other hand p-doping is produced by adding atoms with one less electron in the outer shell compared to silicon. This produces a similar effect in the silicon only now the new levels are close to the valence band (VB) and are empty levels or, as we will refer to there hereafter, filled by a hole. This hole can be thermally excited to the VB at temperatures \textit{T \beq 0K} having a similar effect than the electron promoted to the CB, i.e.: transforming this doped silicon in an electrical conductor.

In making a particle detector both $n$ and $p$ doped silicon are used together by creating the so-called \textit{P-N Junction}. This P-N junction appears when a $p$-doped silicon is put in contact with a $n$-doped silicon, the effects produced by this interface can have an effect on the whole silicon detector if carefully manufactured, as we will see later. In this interface we have too neutral materials with different chemical potentials due to the excess of holes ($p$ silicon) and electrons ($n$ silicon). This chemical potential differences forces the holes to move to the $n$ and the electrons to drift to the $p$ side of the interface. In turn this balancing of the chemical potential means that the $p$ side of the junction gets emptied of holes and filled with electrons effectively turning into a negatively charged space while the $n$ side undergoes the opposite process becoming positively charged.

This rearrangement of the charges generates a potential difference that forces holes and electrons to return to their initial side of the junction. The balancing of the chemical potential and the electrical potential yields a stable state in which an electric field is present on both the $n$ and $p$ sides of the junction.  


\section{Carrier Transport}


\section{Recombination} 

\section{Carrier Generation}

\section{Signal Generation: Ramo's Theorem} % No es Sergio Ramos, es otro Ramo's

