\chapter{TRACS: the meats and potatoes of this whole thing}
\label{chap:Comp}

We have already stated the importance of simulating software in research in general and in radiation damage studies in particular, so it should comes as no surprise that this next chapter is entirely dedicated to the software developed during this project.

In this chapter we will present the structural design of the software as it was before the project started including the different parts in which it is devided. We will also presentthe changes and upgrades performed with particular emphasys on the implementation of radiation damage simulation including the rationale behind the decisions and simplifications made in the process. Lastly we shall comment briefly on where the software stands now, after the upgrades, and what the forseenable future might hold for TRACS from a wider perspective.

\section{Software design} % (fold)
\label{sec:results_and_achievements}

TRACS structure before and after the project (1subsection per component)

here we shall talk about the initial goals and how well we accomplished them. Having experimental and simulated measurements is great (+1pt for the project) but we need to analyse how good the simulator is (+-1pt depending on the outcome).

The project deserves between A+ and B+ => This should be the take away of the conclusion

% section results_and_achievements (end)

\section{The development phase, changes and rationale} % (fold)
\label{sec:let_s_put_the_project_in_perspective}

\subsection{What we changed}

\subsection{How we implemented the stuff we implemented}

It's time to sell the product! It has been presented in RD50 (Hamburg) and validated by the collaboration (at least PdC part). The tool is unique in many ways. From its never-seen-before ability to manipulate and fit the space charge distribution inside a detector to the more common but still in development capacity to simulate and predict transients on radiated silicon detectors.

It's a tool for the future, working in the limits of knowledge and (maybe) pushing them forwards. (toma sobrada!)

% section let_s_put_the_project_in_perspective (end)
