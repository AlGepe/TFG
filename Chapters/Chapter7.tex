\chapter{Conclusions and improvements (maybe)}

The open software application TRACS has been upgraded so that is capable of simulating the transient currents of an irradiated silicon detector. To reproduce the radiation damage in silicon detectors trapping effects as well as three \neff parametrisations (Linear, Triple Constant and Triple Linear) are available to the user. The parameters the define such quantities are left as input values.

Simulations have been timed to take $\sim 30s$ per transient simulated with reasonable simulation values. Simulations performed with TRACS have been compared to laboratory measurements and it was found agreement between both sets of data. The agreement proves that TRACS can be used as a tool to estimate the \neff inside a silicon detector which was the ultimate goal of this project.

TRACS has also been improved with better documentation and more comments throughout the code and with the inclusion of TRACSInterface. This new feature allows for TRACS to be used as a library in any software.  All these improvements together make of TRACS a fast simulator capable of reproducing the effects of radiation in the transient currents of silicon detectors. Future improvements may include the inclusion of TRACS as part of a bigger fitting software that can obtain the \neff distribution of a silicon detector given the transients produced in the lab. For that purpose performance needs to be improved by means of parallelisation.


