\chapter{The effects of radiation in silicon detectors}

We have already seen how particles can leave a trace in silicon detectors in the form of an electrical signal that can be mesaured and analysed. But this particles may also have other effects in the silicon since, provided they have enough energy, they can knock off silicon atoms from their original position in the lattice. This disrupts the silicon lattice therefore inducing changes in its properties and behaviour making it very important to understand this changes to predict and react to this degradation as well as to design radiation-hard detectors for future, more powerful colliders and experiments. 

Radiation damage is a topic of increasing importance in HEP as its effects become increasingly important with increasingly more powerful and higher luminosity colliders. Silicon detectors, due to their ability to measure particle position with very high precision, sit very close to the vertex of the collisions. Taking again the LHC as an example, all the main experiments that operate using its proton-proton collisions use an inner silicon tracker for vertex location. Being the innermost dectectors in the experiments that means they are exposed to bigger amounts of radiation than any other detector in the experiments. Understanding how radiation degrades the performance of such exposed detectors is therefore critical and will only increase its importance in the following years with projects like the high luminosity upgrade of the LHC as well as other future projects such as ILC and CCC currently awaiting approval.

In this section the different mechanisms of radiation damaging will be explained together with their effects on silicon detectors as currently understood. This will set the basis upon which the simulation software TRACS has been built and improved and serve as a justification for the simplifications that needed to be done in the process. The section will also establish the importance of radiation damage studies as well as the current limits of knowledge regarding radiation damage in solid state detectors and the challenges TRACS might be able to help solve in the future.


\section{Wrecking the lattice}

When a particle goes through a silicon detector it can knock off an electron out of the valence band and into the conduction band generating an electrical signal than be read, as we have previously discussed. Another possibility is that the particle knocks off one or more of the silicon atoms that compose the lattice. In this case the atom might return to its original position if the momentum transfer is small dissipating the extra energy via thermal vibrations or it might get knock off completely from its original position. In the latter case the lattice presents now a deffect that will influence its properties.

\section{It's a trap}

\section{Signal degradation}
