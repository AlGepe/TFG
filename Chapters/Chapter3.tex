\chapter{The effects of radiation in silicon detectors}% No definitivo
\label{sec:rad}
% I haven't talked about leakage current and I most certainly should

When a particle goes through a silicon detector, it can loose energy via ionization, which is a reversible process because the electrons ripped from the atoms will end up recombining. The particle can also lose energy via non-ionizing interactions, where the particle interacts with the Silicon atoms or with the dopant atoms. The effect is detrimental for the detector and leads to a reduction of the collected charge, amongst other macroscopical effects. It is important to understand these changes and have a parametrisation of the degradation. Understanding these defects is the fudamental for designing radiation-hard detectors.

In this chapter we summarise the most important mechanisms of radiation damage. One of the key points of the project was for me to implement radiation damage on TRACS simulator.


\section{Wrecking the lattice}% No definitivo

When a particle transversing the silicon detector undergoes a non-ionising interaction one or more atoms get knocked-out from their equilibrium position in the lattice, disrupting the ordered structure of silicon. If the atom does not return to the equilibrium position, and Interstitial-Vacancy (I-V) deffect is created. The term Interstitial refers to the knocked-off atom located now in a position between lattice point and the Vacancy refers to the empty place in the lattice that it leaves behind. This type of deffect alters the ordered structure of the lattice and creates new energy levels previously non-existent. The mechanisms by which the I-V deffects degrade the performance of the detector and its practical consequences will be discussed in the next sections.

%When a particle goes through a silicon detector it can ionise the material and move electrons out of the valence band into the conduction band generating an electrical signal than be read, as we have previously discussed. Another possibility is that the particle knocks off one or more of the silicon atoms that compose the lattice. In this case the atom might return to its original position if the momentum transfer is small dissipating the extra energy via thermal vibrations or it might get knock off completely from its original position. In the latter case the lattice presents now a defect that will influence its properties. For this section we will only focus on the types of defects and their evolution, leaving their effects to the next two sections. 

An inpinning particle can create more than one I-V deffects provided it has enough energy. The second I-V deffects can be created directly by the particle or by the recoil atom if enough energy was transfered to it. The size and clusteting of the defects depends heavily on the energy and type of radiation undergone by the detector. 

%During the radiation process, either in a controlled environment for research purposes or as a side effect of particle detection, various types of particles with different energies might transverse the silicon detector might have different damaging effects in the silicon lattice. When an inpinning particle knock off an atom the result is an interstitial-vacancy pair. The knocked-off atoms sitting in a different position between other silicon atoms is what we call interstitial defect and the lack of such atom on its original position is called vacancy. Depending on the inpinning particle type and the energy transferred to the recoil atom more than vacancy-interstitial defects might be created after the first impact. The size and clustering of these defects depends strongly on the energy of the particles and also on their characteristics. 

For example, photons with up to 1 MeV energies will create only point defects with no clustering effects. Electrons can create both point defects and clusters of defects, with clusters only occurring for electron energies above 8 MeV. Neutrons will create mainly clusters starting at energies as low as 35keV. 

Typically the NEIL (Non-Ionising Energy Loss) hypothesis is used to characterise the radiation damage caused by any type of particle. The NEIL hypothesis states that the damage produced by radiation of any kind of particle is proportional to the damage produced by 1 MeV neutrons. Therefore the standard unit of measuring the radiation that a silicon detector has undergone is the neutron equivalent (neq).

The I-V defects induced by radiation are not necessarily static defects and their size and configuration may vary in time. The damage due to radiation evolve with time. The evolution of defects is not always detrimental; it has been shown that after irradiation detector performance improves with time in a first stage. After the benefitial phase, detector performance only worsens with time. The evolution speed depends heavily on temperature. Evolution achieved in around 21h at 60C can be slowed down to about 500 years just by cooling down the irradiated detector to -10C. 

Typical procedure after irradiation of the detector includes and annealing phase to improve detector performance. From then on, the detector is stored at low temperatures. Such procedure ensures that the detector will be in optimal conditions for laboratory measurements.

%Another important aspect of radiation damage and radiation-induced defects is their evolution in time. It has been shown that the effects of radiation on silicon detector evolve with time even after the irradiation has been stopped. Studies on annealing of irradiated silicon detectors show that typically there is a beneficial evolution right after the irradiation process in which the silicon damage becomes smaller. After this stage annealing increases the damage on irradiated silicon detectors with time reaching levels well over the initial state right after irradiation. This process can be slowed down by lowering the temperature of the detector. Evolution achieved in around 21h at 60C can be slowed down to about 500 years just by cooling down the irradiated detector to -10C. It is for this reason that silicon detectors are usually annealed for a short period of time after irradiation, and the stored at low temperatures for transport and experimental measurements.

\section{Trapping effects} % No definitivo

Defects in the silicon lattice have no effect in the generation of $e-h$ pairs but do have an effect in their drift and collection. This is due to the defects introducing what are called $deep$ levels. Such deep levels lie in the gap between the conduction and valence bands but much further from them than the shallow levels introduced by impurities as explained in section [SECTION].

Deep levels can be filled by electrons or holes depending on their proximity to the valence or conduction band respectively. The main effect of these deep levels is trapping charge carriers for a large amount of time compared with typical collection times in non-irradiated silicon detectors. The trapping of the charge carriers modifies that distribution of electrons and holes inside the detector which in turn can significantly modify the space charge distribution inside the depleted region of the detector. The drift of both charge carriers is also affected by these changes as we will see in detail later. In special cases the effect can be so strong that type inversion might occur. When this happens the previously $p$ doped part of the silicon detector becomes effectively $n$ doped. This effect of type inversion does not happen in initially $n$ doped silicon.

If one looks at the radiation-generated deep levels, they can hold charge carriers for long times before realising it. This process disrupts temporarily the $N_{eff}$ inside the silicon detector. If the trapping-releasing cycle happens with enough frequency (as it is the case under most circumstances) then the \neff can be thought of being permanently modified by the charges trapped in the deep levels. As opposed to the unirradiated case, now the \neff is not a constant but has a different shape. Since the \neff is one of the most important things in signal development inside the silicon detector, it is of great importance that its modification due to radiation damage are well known.

Being, as it still is, a body of great debate, \neff parametrisation has yet to find a definitive answer that can be succesfully applied in all cases. For now we will focus on mainly two parametrisation of said variable for those two are the most succesful and widely accepted, but it should be noted, again, that \neff evolution with radiation damage is not fully understood at the time of writing this work. First of the parametrisation was introduced in 2002 and devised the \neff inside the detector to change shape from constant to linear with depth. In this model the already present electric field inside the detector would force influence charge carriers to move to both ends of the detector (depending on the charge sign) creating an excess of positive charges on one side and a excess of negative ones on the other compared to the non-irradiated state. The resulting \neff is a straight line that gives rise to a parabolic electric field that will change the shape of the collected signal in accordance to experimental results.

The second mayor parametrisation that should also be considered in the frame of this project is a variation on the basic fundamentals of the linear model. This next model was proposed more recently by G.Kramberger and states that the shape of \neff after irradiation can be parametrised in some cases as being constant in three separated regions. As it can be seen in Figure FIGURE\_3ZONE\_NEFF each of the three parts of the \neff would be constant generating a linear electric field with three different slopes resembling the shape of the aforementioned parabola that is observed experimentally.


\section{Signal degradation} % No definitivo

After taking a look at the effects that radiation has on silicon it is important to understand how radiations changes the read-out signal, since that is what will be measured in the laboratory. As we have already mentioned the main effect of radiation damage from a practical point of view is the creation of deep levels inside the band gap that act as trapping centers for the charge carriers moving inside the silicon bulk. For the sake of simplicity and ease of explanation when explanation when talking about the implementation of radiation damage in the software TRACS, we will treat \neff deformation as a different effect. This is very practical as it helps understand dynamical and static effects separately. 

The effect that deep levels have in the signal is the trapping of the charge carriers creating a loss in charge collection efficiency. Because collection times in silicon detectors are of the order of tens of nanoseconds whilst the trapping times are typically of the order of miliseconds, the trapped carriers will never be released in time to be collected and are effectively lost. The process of trapping is a statistical one with the probability of one charge carrier to be trapped after a drifting inside the silicon for a time $t$ being  \[FORMULA FOR TRAPPING PROBS\] where $\tau$ (trapping time) is an experimentally determined parameter that shows how likely it is for a carrier to get trapped, in a very similar law as that governing the natural decay of radioactive materials. For a big enough number of carriers drifting in silicon, the effective result in the signal generated is an exponential decrease over time in the collected signal with respect to the unirradiated case. 

On the other hand the \neff modification due to radiation has no effect on the total collected charge, but on the shape of the collected signal as well as in the collection time. As discussed before the intensity recorded is proportional to the electric field in which the carriers are drifting. Since the electric field is obtained by integrating the \neff, this field is now modified by radiation and yields different waveforms when measured in the lab. Since velocity is different from the non-irradiated case, collection times are modified becoming larger after irradiation in most cases due to lower electric field modulus in the middle of the bulk of silicon. 

Both effects combined yield the typical $double peak$ shape obtained when subjecting the irradiated detectors to the TCT analysis that are normally performed in the lab. This shape combines the double peak feature of the electric field with the exponential dampening of the electrical charge collected. This kind of measurements and analysis are very useful in studying the charateristics of irradiated silicon detectors including determination of trapping times, \neff profile and charge collection efficiency, as it will be shown in the following chapter.






















