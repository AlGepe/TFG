\chapter{The effects of radiation in silicon detectors}

We have already seen how particles can leave a trace in silicon detectors in the form of an electrical signal that can be mesaured and analysed. But this particles may also have other effects in the silicon since, provided they have enough energy, they can knock off silicon atoms from their original position in the lattice. This disrupts the silicon lattice therefore inducing changes in its properties and behaviour making it very important to understand this changes to predict and react to this degradation as well as to design radiation-hard detectors for future, more powerful colliders and experiments. 

In this section the different mechanisms of radiation damaging will be explained together with their effects on silicon detectors as currently understood. This will set the basis upon which the simulation software TRACS has been built and improved and serve as a justification for the simplifications that needed to be done in the process. The section will also stablish the importance of radiation damage studies as well as the current limits of knowledge regarding radiation damage in solid state detectors and the challenges TRACS might be able to help solve in the future.


\section{Wrecking the lattice}

\section{It's a trap}

\section{Signal degradation}
