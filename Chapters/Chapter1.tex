\clearpage
\lhead{\emph{Chapter 1: Introduction}}  % Set the left side page header to "Symbols"
\chapter{Introduction}

There are many kinds of particle detectors in High Energy Physics ranging from the first Bubble and Cloud Chambers up to the latest High Voltage CMOS (Complementary Metal Oxide) solid state detectors. Their design varies depending on many parameters: application, magnetic field, operational temperature, active medium, measurement rate, layout, radiation damage they need to withstand... making it possible to optimise detectors for specific needs. During the past decades, silicon has become the technology of choice for tracking detectors, as it tends to offer an excellent compromise between energy, space and time resolution at a fairly moderate price. Another advantage of silicon detectors is the existing R\&D for other applications, and the possibility to go on commercial technologies.

 The scope of this work is the modelling of the performance of silicon detectors exposed to high fluence of particles, such as those existing in particle accelerators, like the Large Hadron Collider (LHC) at CERN\footnote{European Organization for Nuclear Research, the acronym is derived from its former name in french: Conseil Européen pour la Recherche Nucléaire} and future ones, like its High Luminosity upgrade(HL-LHC) . There, the innermost tracking detectors will have to withstand particle fluences up to $10^{16}$ $n_{eq}/cm^{2}$  during 10 years of operational lifetime. The amount of collected charge will decrease over this period due to trapping of charge carriers in radiation induced traps. The detectors will have to be designed such that the charge loss is above an electronics threshold of $\approx$ 6000 $e^{-}$, which is the minimum the electronics can resolve. 

 Understanding the effects of radiation in Si is a crucial task. These detectors have to operate for a period of 10 years without being replaced. In this project, a model of radiation damage in silicon has been implemented into an already existent simulation software called TRACS\cite{TRACS}. 
 
In Chapters \ref{chap:detector} and \ref{chap:rad} we will introduce some background information on how silicon detectors work and how radiation affects their internal structure and performance. Chapter \ref{chap:TCT} will describe an experimental technique used in this work to characterise silicon detectors. This data will serve as an input to crosscheck the results of the simulation. In chapter \ref{chap:tracs} the inner workings of the software and the actual implementation of radiation damage is presented. Finally, chapter \ref{chap:TRACSvalidity} will show a comparison between simulation and experimental data to prove the validity of the simulations. 

%\section{Solid State Detectors in High Energy Physics}
%
%In this section we shall clarify how and why particles are detected using solid state detectors in particular Si. 
%
%\subsection{Principles of particle detection}
%
%It should be clear how particles are detected in almost every single detector existing today: Have detector with Electric field inside; particle goes boom; boom creates charges; charges drift in electric field; drift generates signal; signal tells how is the particle that went boom. This could be a good place to introduce concepts such as 
%
%\subsection{Specific charateristics of a \textit{Si}-based particle detector}
%
%In this subsection we should tackle everything specific to \textit{Si} detectors, from te advantages over gas detectors to the technical details of their design and operation. Typical numbers can be helpful. The Diode should be explained in detail and its importance in research and understanding should be clear in a context where nobody actually detects particles with them anymore. Should we explain detailed the semiconductors and p-n junction?
%
%\section{Radiation damage in \textit{Si} detectors}
%
%We should clearly explain the effects of radiation in \textit{Si} without going too deep in the differences between proton-like and neutron-like. We can only afford to go deep into traps and trapping as well as space charge modification inside the detector. Leakage current, higher vdep, etc. should be mentioned as wel.
