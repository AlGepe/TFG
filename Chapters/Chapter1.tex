\chapter{Introduction}

Particle detectors are by far the most important tool in High Energy Physics (HEP) being the main (and mostly the only) source of empirical data. There are multiple kinds of particle detectors ranging from the primitive Bubble and Cloud Chambers up to the more sophisticated ones such as the Ring Image Cherenkov gaseous detectors or the High Voltage CMOS solid state detectors. The properties of the detectors vary depending on the materials used and the design layout making it possible to optimize said detectors for specific needs. During the past decades, however, silicon has become one of the most widespreads materials for building detectors as it tends to offer a good compromise between energy, space and time resolution at a fairly moderate price. Another advantage of silicon detectors is the relatively simple design need to achieve good performance, making it easier to understandand manufacture now that the technology has reached great control and precision in the manufacturing process.

Is because of this reasons that Silicon detectors are widely used not only in HEP but in many more research fields and even in every-day life. Nevertheless, as HEP evolves new need arise and new problems have to be considered. In this project we will focus on the degradation process that silicon detectors undergo when exposed to high amounts of radiation such as those found in big particle accelerators like the Large Hadron Collider (LHC) at CERN\footnote{Centre Europene pour la Recherche Nucleaire} where the innermost detectors are made or silicon and have to operate for year with high precission and efficiency. 

For this detectors to perfom according to specificaton for their whole lifetime is important that they can receive big amounts of radiation without loss of efficiency. This becomes increasingly important as accelerators grow more powerfull and have higher collision rates. Focusing on the particular example of LHC one important it is now to understand how detectors currently use are dgrating over time and identify when they might need replacement, but it is critical to completely understand the effects of radiation in these silicon detectors so that detectors used in the High Luminosity-LHC (HL-LHC) are designed in such a way that they can have a longer lifetime compared to current ones and maintain a very high efficiency and resolutions even under the heavy radiation fluxes expected for the upgrade in 10 years time.

Now that the importance of studying the effects of radiation in silicon detectors has been clarified it's time to see how this can be done. As in any research quest there are always to main approach that should benefit from one another to give the best results: the theoretical approach, and the experimental one. This project is mainly and experimental project but relys heavily in the known theory of radiation damage in silicon detector as we will see in the following chapters. This project revolves around the idea of creating a tool that helps predict, model and understand said effects as a first step into designing radiation-hard detectors for future utilization in high collision rate accelerator such as the aforementioned HL-LHC.

In particular the subject of this project will be to implement a model of radiation damage in silicon detectors into an already existent simulation software. For such a job it is needed to have a good understanding of how silicon detectors work and how radiation affects its internal structure as well as its performance; this will be explained in the next chapter. Following this basic knowledge about silicon detector behaviour we will dig deeper, in the second chapter, in the current techniques employed to characterize silicon detectors and observe their main properties such as laser illumination in TCT, e-TCT or TPA configurations. This will bring the theoretical basis to and end and will lead to the fourth chapter in which the inner workings of the software and a detailed explanation of how radiation damage was implemented, will be presented. Following this chapter we will present a comparison between simulation and experimental data as proof of the validity of the simulations. To finish this project conclusions will be presented as well as a list of possible future improvements and aplications of such a simulation software.

%\section{Solid State Detectors in High Energy Physics}
%
%In this section we shall clarify how and why particles are detected using solid state detectors in particular Si. 
%
%\subsection{Principles of particle detection}
%
%It should be clear how particles are detected in almost every single detector existing today: Have detector with Electric field inside; particle goes boom; boom creates charges; charges drift in electric field; drift generates signal; signal tells how is the particle that went boom. This could be a good place to introduce concepts such as 
%
%\subsection{Specific charateristics of a \textit{Si}-based particle detector}
%
%In this subsection we should tackle everything specific to \textit{Si} detectors, from te advantages over gas detectors to the technical details of their design and operation. Typical numbers can be helpful. The Diode should be explained in detail and its importance in research and understanding should be clear in a context where nobody actually detects particles with them anymore. Should we explain detailed the semiconductors and p-n junction?
%
%\section{Radiation damage in \textit{Si} detectors}
%
%We should clearly explain the effects of radiation in \textit{Si} without going too deep in the differences between proton-like and neutron-like. We can only afford to go deep into traps and trapping as well as space charge modification inside the detector. Leakage current, higher vdep, etc. should be mentioned as wel.
